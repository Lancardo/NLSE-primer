\chapter{推荐资料}
\label{ch:material}
以下是关于光学、电子学和通信领域的有用且易获取的资源列表。

\section{光学}

\subsection*{光子学要素}
这本由伊豆香敬教授编写的教科书,介绍了实际光学器件背后的理论,涵盖了广泛的话题。特别推荐其对掺铒光纤放大器(EDFA)增益和噪声特性的讨论,EDFA是现代光学和电信中不可或缺的工具。

\subsection*{现代光学讲座,帕尔塔·罗伊·乔杜里教授}
这系列\href{https://www.youtube.com/watch?v=2WiMeh1Dxl8&list=PLbRMhDVUMngePMuAGeAUeGVuZffTFY-5i}{60个免费视频讲座}涵盖了麦克斯韦方程、偏振和双折射等基础内容,还包括波导和电光调制器等器件的更高级主题。强烈推荐,因其结构良好且数学推导严谨,涵盖了所有描述的主题。

\subsection*{非线性光学讲座,萨穆德拉·罗伊教授}
这系列\href{https://www.youtube.com/watch?v=EiIDScj124Q&list=PLbRMhDVUMngfBwyonVP8VIsabtnsV3GVv}{60个免费视频讲座}首先讲解了线性光学中的基本概念,然后从麦克斯韦方程出发,详细推导了二阶和三阶非线性效应。讲解了二次谐波生成(SHG)、差频生成(DFG)、光学参量振荡器(OPO)以及一些超出本教材范畴的现象。特别推荐其对$\chi^{(2)}$和$\chi^{(3)}$非线性对称性和张量性质的解释。

\subsection*{非线性光纤光学}
这本由戈温德·P·阿格拉瓦尔编写的教科书详细解释了方程~\ref{eq:GNLSE}背后的数学原理,并且频繁参考实验结果。强烈推荐其对涉及偏振、受激布里渊散射和新型光纤的高级现象的深入讨论。

\subsection*{麻省理工学院“激光与光学演示”讲座系列}
由沙乌尔·以色基尔教授主讲的\href{https://www.youtube.com/watch?v=1cEXNLP5uE0&list=PL4E7FAAD67B171EBC}{49个免费视频讲座},包含了光学基本效应的实践演示,如偏振、干涉和衍射,演示内容基于自由空间光学。强烈推荐,因其系统的实验设计。

\subsection*{国际参数非线性光学学校讲座}
这系列\href{https://www.youtube.com/@ispnlo9041/videos}{44个免费视频讲座}由非线性光学领域的专家主讲。强烈推荐,因其涵盖了通常在普通课程中未涉及的高级主题。

\subsection*{Les' 实验室}
这个\href{https://www.youtube.com/@LesLaboratory/videos}{YouTube频道}专注于激光技术的实际演示,包括染料激光器、光谱仪、电光调制器、SHG以及超连续谱生成等。强烈推荐,因其对相关效应的直观演示和对操作光学器件所需电子学的详细解释。

\subsection*{YourFavouriteTA}
这个\href{https://www.youtube.com/@yourfavouriteta/videos}{YouTube频道}由本教材的作者主讲,通过实际实验、理论推导和数值模拟解释非线性光学。该频道旨在通过直观的解释,帮助观众理解通常由复杂方程描述的线性和非线性过程。

\section{电子学}

\subsection*{The Signal Path}
这个\href{https://www.youtube.com/@Thesignalpath/videos}{YouTube频道}专注于电源、信号发生器、探测器、示波器、频谱分析仪和其他常见的科学和工业实验室设备的实际实验、拆解和维修。特别推荐其关于射频(RF)设备的视频。

\subsection*{EMPossible}
这个\href{https://www.youtube.com/@empossible1577/playlists}{YouTube频道}聚焦于电磁理论的基础以及射频波导、半导体能带结构和有限元分析等高级话题。强烈推荐,因其覆盖了广泛的概念,并提供了优秀的插图和精简的教程。

\subsection*{Keysight 电子学教程}
Keysight电子设备公司的\href{https://www.youtube.com/@KeysightLabs/videos}{YouTube频道},包含了针对电信号测量的解释。特别推荐,因为该频道专注于示波器,这对于通过光电二极管测量功率随时间变化并可视化是常用工具。

\subsection*{Rohde\&Schwarz 电子学教程}
这个由电子设备公司Rohde\&Schwarz制作的\href{https://www.youtube.com/watch?v=rUDMo7hwihs&list=PLKxVoO5jUTlvsVtDcqrVn0ybqBVlLj2z8}{YouTube播放列表}提供了关于电气设备和信号的测量和表征的深入教程,涵盖了功率、相位噪声、放大器噪声因子等内容。许多这些概念可以很好地转移到光学领域。

\section{通信}

\subsection*{Ian Explains}
这个\href{https://www.youtube.com/@iain_explains/videos}{YouTube频道}解释了数字和模拟信号处理、通信和统计学的基本方法。强烈推荐,因其直观的示例和简明的推导。

\section{仿真工具}

\subsection*{Octave 光子学}
这个免费的互动\href{https://www.octavephotonics.com/nlse}{基于浏览器的工具}用于求解方程~\ref{eq:GNLSE},非常适合可视化色散、孤子形成和拉曼效应等基本现象的影响。强烈推荐,因其简洁易用的界面。

\subsection*{ssfm\_functions.py}
这个\href{https://github.com/OleKrarup123/NLSE-vector-solver/tree/main}{开源Python库}用于求解方程~\ref{eq:GNLSE},由本教材的作者创建并维护。它提供了高度的灵活性和可定制性,能够使用代码设置仿真。例如,可以使用for循环创建由多个具有不同属性(如不同拉曼模型、输入/输出增益或色散补偿)的光纤链路,仿真结果可以使用内建的绘图函数轻松可视化。该工具的可修改性使其非常适合非线性光学领域的研究生,尤其是当他们需要建模新型且高度特定的系统时。

\subsection*{gnlse-python}
这个\href{https://github.com/WUST-FOG/gnlse-python}{开源Python库}用于求解方程~\ref{eq:GNLSE},提供了出色的\href{https://gnlse.readthedocs.io/en/latest/index.html}{文档}和内建的可视化功能。有关实现的arXiv预印本也可用~\cite{redman2021gnlsepythonopensourcesoftware}。

\subsection*{GMMNLSE-Solver}
这个\href{https://github.com/WiseLabAEP/GMMNLSE-Solver-FINAL}{开源MATLAB库}允许求解方程~\ref{eq:GNLSE}的多模版本,并能模拟诸如多模孤子等异乎寻常的现象。它是一个高级工具,推荐用于极限应用,尤其适合已经非常熟悉本教材中所介绍的基本单模情况的用户。如果配置正确,它可以将计算并行化到GPU上,对于如此复杂的仿真来说,速度非常快。
