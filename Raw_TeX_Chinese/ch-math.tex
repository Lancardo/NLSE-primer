\chapter{数学和理论}
\label{ch:MathAndTheory}

本章介绍描述电磁波和理解方程~\ref{eq:GNLSE} 所需的基本数学工具。

\section{实数场和复数场}
%Explain that the real field is what determines how an electron moves and that the complex one is only used for mathematical convenience when dealing with phase shifts.
带电荷$q$、质量$m$的粒子在电场 $\Bar{\E}_r = \E_r \hat{x}$ 作用下的加速度为
\begin{align}
    \Bar{a} &=  \frac{q\E_r}{m}\hat{x}.
\end{align}
电场下标$r$表示这是“实”电场,它决定了带电粒子的加速度。而“复”电场则是一种数学工具,使得涉及电磁波的计算更为简便,并且可以始终从中恢复出实电场。例如,传播在体介质中的实电场波,其空间角频率 $\betag$ 依赖于时间角频率 $\omega$,可以表示为:\begin{align}
\label{eq:real_field}
    \E_r(z,t) &= |\E_0|\cos\left(\betag(\omega)z-\omega t+\phi\right) \\\nonumber 
     &=|\E_0| \real\left\{  \exp\left( i\betag(\omega)z-i\omega t+i\phi \right) \right\} \\ \nonumber 
     &= \real\left\{ \E_0 \exp\left( i\betag(\omega)z-i\omega t\right) \right\}  \\ \nonumber
     &= \real\left\{ \E(z,t) \right\}  \\ \nonumber
     &=  \frac{1}{2}\left( \E(z,t) + \E^*(z,t) \right).   
\end{align}
关于方程 \ref{eq:real_field} 的说明,请参见 \href{https://www.desmos.com/calculator/fgvozursrl}{此交互式图表}。
使用复电场在模拟相位变化和干涉效应时非常方便。例如,要在实际电场中引入相位移 $\phi_0$,必须通过以下方式“手动插入”:\begin{align}
\label{eq:insert_phase}
    \E_r(z,t) &= |\E_0|\cos\left(\betag(\omega)z-\omega t+\phi\right) \Rightarrow \\ \nonumber \E_r'(z,t) &= |\E_0|\cos\left(\betag(\omega)z-\omega t+\phi+\phi_0^{\textcolor{Red}{\swarrow}}\right).   
\end{align}
使用复电场,相同的操作可以通过复数乘法来完成,因为
\begin{align}
\label{eq:insert_phase_complex}
    \E(z,t) &=\E_0 \exp\left( i\betag(\omega)z-i\omega t+ i\phi\right)\Rightarrow \\ \nonumber \E'(z,t) &=  \E_0 \exp\left( i\betag(\omega)z-i\omega t +i\phi\right)\exp\left( i\phi_0\right) \\ \nonumber
    &=\E_0 \exp\left( i\betag(\omega)z-i\omega t +i\phi+ i\phi_0\right) \\ \nonumber
    \E_r'(z,t) &= \real\left\{  \E'(z,t)   \right\}. 
\end{align}

此外,由于振荡的实电场的瞬时功率与其平方成正比,平均功率可以通过积分计算或通过复电场绝对值平方的一半来计算,
\begin{align}
\label{eq:average_power}
    \langle \E_r^2 \rangle_T&= \frac{1}{T}\int_{0}^{T} \E_r^2 dt = \langle\real\left\{\E\right\}^2\rangle_T =\frac{1}{4}\langle\left(\E+\E^*\right)^2\rangle_T\\ \nonumber
&=\frac{1}{4}\langle \E^2+\E^{*2}+2|\E|^2 \rangle_T=\frac{1}{2} \langle|\E|^2\rangle_T=\frac{1}{2}|\E|^2.
\end{align}
使用复电场替代“手动插入”和积分的便利性足以使得采用复电场进行计算并仅在必要时提取实电场变得值得。因此,本简介主要利用复电场,但强调这些仅是有用的数学抽象,而实电场具有物理意义,因为它们直接决定了电荷的加速。
\subsection{实际电场和电场包络}
无线电信号用于Wi-Fi的载波频率约为5 GHz,而最先进的示波器可以测量频率高达100 GHz的电场。相比之下,激光脉冲的电场通常在100 THz以上的载波频率下振荡。因此,以实际复电场 \(\E(z,t) = \E_0 \exp\left( i\betag(\omega_0)z - i\omega_0 t\right)\) 进行计算和表达结果往往不够方便,因为以 \(\omega_0/2\pi\) 次每秒发生的快速电场振荡实际上是无法检测到的。相反,我们可以将复电场的包络定义为
\begin{align}
\label{eq:envelope}
    \A(z,t) &=a\cdot\E(z,t)\cdot e^{-i(\betag(\omega_0)z-\omega_0t)},
\end{align}
并用它进行计算。在这里,\( a = \sqrt{0.5 \epsilon_0 n c A_{eff}} \),其中 \(\epsilon_0\) 是真空的介电常数,\( n \) 是介质的折射率,\( c \) 是光速,\( A_{eff} \) 是光场横截面的有效面积,是一个归一化常数。将 \(\E\) 按 \( a \) 缩放确保 \(\A\) 具有 \(\sqrt{W}\) 的单位,因此 \( |\A|^2 \) 具有 \( W \) 的单位。通过“提取”快速且不可检测但又是\emph{可预测的}时间和空间振荡,确定电场由于线性和非线性效应而\emph{变化}的方式变得更加简单。有关 \(\E\) 和 \(\A\) 之间差异的示例,请参见此 \href{https://www.desmos.com/calculator/rsw2fn5af6}{交互式图表}。


\section{傅立叶变换}
%Define the Fourier Transform so going from time to frequency and back is well-behaved. 

在本教程中,傅里叶变换及其逆变换被定义为:
\begin{align}
    \Tilde{\E}(z,\omega) &= \FT\left\{\E(z,t)\right\} = \int_{-\infty}^{\infty} \E(z,t) e^{i\omega t} dt, \\ \nonumber
    \E(z,t) &= \IFT\left\{\Tilde{\E}(z,\omega)\right\} = \frac{1}{2\pi} \int_{-\infty}^{\infty} \Tilde{\E}(z,\omega) e^{-i\omega t} d\omega.
\end{align}
使用 $\exp(i\omega t)$ 的傅里叶变换约定,而不是 $\exp(-i\omega t)$,是因为沿着正 z 方向传播的复平面波由 $\exp(i\betag(\omega_0)z-i\omega_0 t)$ 描述。因此,计算,

\begin{align}
    \FT\left\{\exp(i\betag(\omega_0)z-i\omega_0 t)\right\} &= \int_{-\infty}^{\infty} e^{i\betag(\omega_0)z-i\omega_0 t} e^{i\omega t} dt, \\ \nonumber
      &= \int_{-\infty}^{\infty} e^{i\betag(\omega_0)z-i(\omega_0-\omega) t} dt \\ \nonumber
      &= e^{i\betag(\omega_0)z}\delta(\omega-\omega_0),
\end{align} 
显示复平面波的傅里叶变换产生一个以正载波角频率 $\omega$ 为中心的 δ 函数。如果在傅里叶变换中使用 $\exp(-i\omega t)$ 并将其应用于沿正 z 方向传播的复平面波,结果将包含 $\delta(\omega_0+\omega)$,这意味着 δ 函数是以负载波角频率为中心的。后一种方法使得在涉及沿 z 方向传播的复平面波的傅里叶变换的计算中更复杂,因此采用前一种约定。

\section{脉冲}
在介质中具有有限持续时间的电磁脉冲可以视为一组不同的复平面波的无限和,如下所示:
\begin{align}
    \label{eq:pulse}
    \E(z,t) &= \frac{1}{2\pi}\int_{-\infty}^{\infty} |\Tilde{\E}(z,\omega)| e^{i\betag(z,\omega)z-i\omega t+i\phi(z,\omega)} d\omega \\ \nonumber
    \E(z,t) &= \frac{1}{2\pi}\int_{-\infty}^{\infty} \Tilde{\E}(z,\omega) e^{i\betag(z,\omega)z-i\omega t} d\omega.
\end{align}
方程~\ref{eq:pulse} 提供的关键见解是,光信号的强度、形状和颜色的任何变化都必须源于改变一组时间频率分量的幅度 $|\Tilde{\E}(z,\omega)|$、相位 $\phi(z,\omega)$ 或空间频率 $\betag(z,\omega)$。即使是非线性效应,这些效应可以以令人惊讶的方式改变激光脉冲,本质上也只不过是改变这三个参数。


\section{啁啾和延迟}
%Highlight that how the phase changes with time determines instantaneous frequency, which is super important for understanding pulse evolution. Explain that the change in phase w.r.t. frequency determines the time shift for each frequency.
考虑在 $z=0$ 处给出的复电场

\begin{align}
\label{eq:chirp_example}
    a\E(t) &= \A(t)\exp\left(  -i\left(\omega_0 +\frac{C}{2}T \right)T   \right).
\end{align}

如果 $C=0$,电场的相位线性变化,这意味着它以固定的载波频率 $\omega_0$ 振荡。如果 $C>0$,方程~\ref{eq:chirp_example} 表明载波频率会随时间增加,而 $C<0$ 则意味着载波频率会降低。请参见 \href{https://www.desmos.com/calculator/gd7s8nhfdn}{这个互动图} 以获取说明。电场的“瞬时角频率”定义为

\begin{align}
\label{eq:chirp_definition}
    \delta\omega(T) &= -\partial_T\phi(T),
\end{align}

其中 $\phi(T)$ 是电场的相位随时间的函数。负号在时间导数前面是为了确保沿 z 方向传播的复平面波的瞬时角频率被正确计算为 $+\omega_0$,其表达式为 $\exp(i\beta(\omega_0)z-i\omega_0 t)$。

瞬时频率随时间变化的电场被称为“啁啾”。在频率低于(高于)其载波频率的时间段称为“红啁啾”(“蓝啁啾”)。从“红色”变为“蓝色”的啁啾称为“增加”,而从“蓝色”变为“红色”的啁啾称为“减少”。

理解已知电场的不同持续时间可以具有不同的瞬时频率,并且这些频率可以通过计算相位的负导数从方程~\ref{eq:chirp_definition} 中获得,对于理解许多线性和非线性效应至关重要。例如,如果某一介质中的光速使得高频(即“更蓝”的频率)比低频(即“更红”的频率)传播得更快,那么通过这种材料传播的光脉冲将在前面形成蓝啁啾,而在后面形成红啁啾。

正如对时间相位的导数可以提供有关瞬时频率的信息一样,可以对光谱相位关于频率的导数进行计算,以确定特定频率的时间延迟。考虑在 $z=0$ 处给出的复电场包络的光谱

\begin{align}
\label{eq:spectrum_time_example}
    \Tilde{\A}(\omega) &= \Tilde{\A}_0(\omega)\exp\left( i\left(\frac{B}{2}(\omega-\omega_0)^2 \right)   \right).
\end{align}

假设 $B>0$,方程~\ref{eq:spectrum_time_example} 表明围绕载波角频率 $\omega_0$ 的频率分量的相位随着与载波的距离增加而以二次形式增加。或者,可以将方程~\ref{eq:spectrum_time_example} 解释为:对于 $\omega_0$ 以下的角频率,相位 \emph{减少},而对于 $\omega_0$ 以上的角频率,相位 \emph{增加}。受方程~\ref{eq:chirp_definition} 启发,我们可以计算

\begin{align}
\label{eq:delay_definition}
    \delta t(\omega) &= \partial_\omega\phi(\omega),
\end{align}

对于方程~\ref{eq:spectrum_time_example},得出

\begin{align}
    \delta t(\omega) &=  B(\omega-\omega_0),
\end{align}

这表明 $\omega_0$ 以下的角频率经历负时间延迟(使其更早到达),而 $\omega_0$ 以上的角频率经历正时间延迟(相当于延迟)。请注意,在方程~\ref{eq:delay_definition} 中没有需要改变符号的要求以保持一致。

与方程~\ref{eq:chirp_definition} 相比,方程~\ref{eq:delay_definition} 对于计算的实用性较差,但有关相位随角频率的降低意味着提前到达时间,而相位随角频率的增加意味着延迟到达的见解在分析第~\ref{ch:Dispersion} 章中色散的影响时是有帮助的。有关改变光谱相位与时间延迟之间关系的说明,请参见 \href{https://youtu.be/E3S0BQiy3p8}{此视频教程}。





